% Template FAQ del network stataleinformatica

\documentclass{article}

\usepackage[utf8]{inputenc}
\usepackage[T1]{fontenc} 
\usepackage{hyperref}
\usepackage[dvipsnames]{color}
\usepackage[nodayofweek,ddmmyyyy]{datetime}
\usepackage{enumitem}

\newcommand{\domanda}[1]{\item{\ttfamily{\textsl{{\large{#1}}}}\\\\}}


\title{FAQ - \textbf{Complementi di algoritmi e strutture dati}}

\author{
	Corso complementare \\6 CFU, 2$^{\circ}$ Semestre\\
	Docente: Nicolò Cesa-Bianchi \\ 
	\href{http://cesa-bianchi.di.unimi.it/Algo2/}{Sito web}
	\date{}
}

\begin{document} 
\maketitle

\begin{enumerate}

\rmfamily
\domanda{Dove si trova il sito web del corso?} 
    Si trova qui sopra se clicchi "sito web".
\\
	
\domanda{Il corso è ancora attivo?} 
    Si, ma lo è ad anni alterni: verrà erogato nell'anno solare 2021, poi ancora nel 2023 e cosi' via.\\
    Banalmente, significa che chi ora è al secondo o terzo anno deve necessariamente seguire il corso nel 2021, se vuole frequentare e laurearsi in tempo.
\\

\domanda{Come è strutturato l'esame?}
    L'esame consiste in una singola prova orale, fissata su appuntamento (eventualmente anche fuori sessione).
    \\\\
    Si tratta di un orale di circa 20-30 minuti che verte su tutti gli argomenti del corso, con particolare attenzione alle classi di complessità computazionale e agli algoritmi probabilistici. Essendo un esame orale, la modalità rimane inalterata anche da remoto (usando zoom).
    \\\\
    Solitamente, la prima domanda del professore richiede una risposta dettagliata con tanto di dimostrazione scritta, mentre le successive 2-3 domande, pur richiedendo comunque un certo rigore, possono essere affrontante a voce e con un po' più di \textit{hand waving}.
    \\\\

Esempi di domande possono essere trovate cercando nella chat.
\\
\domanda{Qual è il materiale a disposizione per studiare?}
    Il docente mette a disposizione degli ottimi appunti sulla \href{http://cesa-bianchi.di.unimi.it/Algo2/}{homepage del corso}, che sono esattamente gli appunti su cui si basa per le lezioni. Sul sito sono anche indicati i libri di testo consigliati.\\
    Inoltre, nella risposta successiva è indicato del materiale supplementare, tra cui delle registrazioni audio delle lezioni 2018/2019.
\\
\domanda{Quali sono le risorse consigliate?}
\begin{itemize}
		\item \href{https://mega.nz/#F!4xd2gC7A!AEZhvQBEIl2ipcyHpFN-JA}{Registrazioni audio delle lezioni 2018/2019}
		
		\item Il libro KT per la prima parte del programma (Cenni alla complessità computazionale)
		
		\item \href{https://www.youtube.com/watch?v=KqMGeNZuwfI}{Serie di video di Paul Learns Things} per l'algoritmo di Karger
		
		\item \href{https://www.youtube.com/watch?v=z0lJ2k0sl1g}{Lezione del MIT} per l'hashing universale
\end{itemize}

\end{enumerate}
\end{document}

