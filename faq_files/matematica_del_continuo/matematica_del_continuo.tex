\documentclass{article}

\usepackage[utf8]{inputenc}
\usepackage[T1]{fontenc} 
\usepackage{hyperref}
\usepackage[dvipsnames]{color}
\usepackage[nodayofweek,ddmmyyyy]{datetime}
\usepackage{enumitem}

\newcommand{\domanda}[1]{\item{\ttfamily{\textsl{{\large{#1}}}}\\\\}}


\title{FAQ - \textbf{Matematica del continuo}}
\author{
	1$^{\circ}$ anno\\12 CFU, Primo semestre\\
	Docenti: Cecilia Cavaterra, Anna Gori\\ 
	\href{https://ccavaterramc.ariel.ctu.unimi.it/v5/home/Default.aspx}{Sito web}
	\date{}
}

\begin{document} 
	\maketitle
	
	\begin{enumerate}
		
		\rmfamily
		
		\domanda{Dove si trova il sito web del corso?} 
		Si trova qui sopra se clicchi "sito web".\\
		
		\domanda{Come è strutturato l'esame?}
		L'esame consiste in una prova scritta e in una prova orale da svolgersi nello stesso appello. 
		La prova scritta dura 2 ore ed è formata da due parti distinte. 
		Piu info su come sono composte le due parti sull'ariel del corso.\\
		
		\domanda{Qual è il materiale a disposizione per studiare?}
		Sono disponibili le videolezioni con le relative slide, e anche delle esercitazioni.
		Il libro consigliato dalla docente è: "P. Marcellini e C. Sbordone, Calcolo, Liguori".\\
		
		\domanda{Quali sono le risorse consigliate?}
		- Wolfram Alpha (https://www.wolframalpha.com/)
		
		- Geogebra (https://www.geogebra.org/graphing)
		
		- Videolezioni (http://vc.di.unimi.it/)
		
		- Stampati lezioni del Prof. Gobbino (https://goo.gl/9ADmUc)
		
	\end{enumerate}
	
\end{document}
