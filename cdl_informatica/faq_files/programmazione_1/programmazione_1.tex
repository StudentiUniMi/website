\documentclass{article}

\usepackage[utf8]{inputenc}
\usepackage[T1]{fontenc} 
\usepackage{hyperref}
\usepackage[dvipsnames]{color}
\usepackage[nodayofweek,ddmmyyyy]{datetime}
\usepackage{enumitem}

\newcommand{\domanda}[1]{\item{\ttfamily{\textsl{{\large{#1}}}}\\\\}}


\title{FAQ - \textbf{Programmazione I}}
\author{
	1$^{\circ}$ anno\\12 CFU, Primo semestre\\
	\date{}
}

\begin{document} 
	\maketitle
	
	\begin{QuestionList}
		
		% TODO: Fa schifo, aggiungere i link giusti
		\question{Dove si trovano i siti web del corso?} {
		    È possibile trovare gli avvisi, il programma e altre informazioni riguardo il corso al seguente \href{http://boldi.di.unimi.it/Corsi/Inf2020/}{link}.
		
		    Sito per i cognomi da A a L:
	        \href{https://atrentinip.ariel.ctu.unimi.it/}{link}
	        
	    	Sito per i cognomi da M a Z:
		    \href{https://mcasazzapud.ariel.ctu.unimi.it/}{link}
		}
		
		\question{Come è strutturato l'esame?} {
		    L’esame è composto da una prova di programmazione individuale in laboratorio (la prova contiene un esercizio di filtro: gli studenti che non superino il filtro non saranno poi valutati) e una prova scritta. 
		    
		    A chi supera entrambe le prove viene proposto un voto (ottenuto come media dei voti delle due prove). Il voto può essere
		    \begin{itemize} 
		        \item <= 20: \textbf{devono} sostenere un esame orale di "controllo" per verificare le competenze, generalmente basato sulla risoluzione in "diretta" di uno o due esercizi di programmazione
		        
		        \item 21 <= x <= 27: possono decidere se accettare o rifiutare il voto proposto (rifiutando, si avrà la possiblità di sostenere nuovamente l'esame al prossimo appello utile). Non è possibile sostenere l'orale per alzare il voto.
		        
		        \item >= 28: \textbf{possono} verbalizzare il voto proposto o scegliere di sostenere un esame orale (finalizzato a migliorare il voto)
		    
		    \end{itemize}
		   
		    Le varie parti da cui l'esame è composto vanno necessariamente sostenute nello stesso appello, e in particolare chi pur avendone la possibilità decide di non presentarsi alle verbalizzazioni dovrà sostenere nuovamente l'esame.
		}
		
		\question{Qual è il materiale a disposizione per studiare?} {
		    I docenti consigliano i seguenti libri:
		    \begin{itemize}
		        \item Ivo Balbaert: Programmare in go. Pearson, ISBN 8891909661.\\
		        \item Alan A. Donovan, Brian W. Kernighan: The Go Programming Language, Addison-Wesley.\\
		    \end{itemize}
		    
		    Per esercitarsi, sono inoltre disponibili i vecchi temi d'esame sul \href{http://boldi.di.unimi.it/Corsi/Inf2020/}{sito del Prof. Boldi} (controlla gli anni accademici precedenti), e gli \href{https://labprog.mapio.it/}{esercizi del Prof. Santini} (con commenti e test)
		}
		
	\end{QuestionList}
	
\end{document}
