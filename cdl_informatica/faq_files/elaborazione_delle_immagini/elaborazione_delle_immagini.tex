\documentclass{article}

\usepackage[utf8]{inputenc}
\usepackage[T1]{fontenc} 
\usepackage{hyperref}
\usepackage[dvipsnames]{color}
\usepackage[nodayofweek,ddmmyyyy]{datetime}
\usepackage{enumitem}

\newcommand{\domanda}[1]{\item{\ttfamily{\textsl{{\large{#1}}}}\\\\}}


\title{FAQ - \textbf{Elaborazione delle immagini}}
\author{
	Corso complementare\\6 CFU, Primo semestre\\
	Docente: Raffaella Lanzarotti\\ 
	\date{}
}

\begin{document} 

\maketitle
	
\begin{QuestionList}
		
    \question{Dove si trova il sito web del corso?} {
    	Il corso ha due siti web principali: 
    	\begin{itemize}
    	\item l'Ariel del corso di Informazione Multimediale, su cui trovare comunicazioni e videolezioni, disponibile
    	al seguente link:\\ \href{https://ggrossiim.ariel.ctu.unimi.it/v5/Home/}{https://ggrossiim.ariel.ctu.unimi.it/v5/Home/}
    	\item Il sito Labonline con i materiali interattivi per le lezioni, disponibile al seguente link:\\
    	\href{https://labonline.ctu.unimi.it/course/view.php?id=180}{https://labonline.ctu.unimi.it/course/view.php?id=180}
        \end{itemize}
	}
		
	\question{Come è strutturato l’esame?} {
		L'esame consiste di due parti:
		\begin{itemize}
		\item una prova scritta con domande aperte su tutti gli argomenti affrontati a lezione (pesa per il 75\% del voto finale);
		\item una prova pratica di programmazione Python (pesa per il 25\% del voto finale).
		\end{itemize}
		E' prevista una prova al termine delle lezioni (circa a fine Novembre) con modalità analoghe alle prove d'appello (scritto più prova pratica).
	}
		

		
	\question{Qual è il materiale a disposizione per studiare?} {
		Sono disponibili videolezioni sull'Ariel del corso di Informazione Multimediale e quiz e materiale interattivo di laboratorio sul Labonline (vedi sezione 1).
		Non è presente un testo di riferimento al di fuori dei materiali forniti.
	}

\newpage

	\question{Quali sono le risorse consigliate?}{
		\begin{itemize}
		\item Il Colab accessibile tramite le esercitazioni di Labonline
		\item Python con le librerie OpenCV, Numpy, Plotly, Matplotlib
		\item Le videolezioni disponibili su Ariel
		\end{itemize}
	}

	\question{Posso iscrivermi all'esame di Informazione Multimediale avendo seguito solo Elaborazione delle immagini?}{
		No. L'esame di elaborazione delle immagini è solo una parte di tre insegnamenti totali necessari per iscriversi all'esame di Informazione Multimediale. Inoltre, Informazione Multimediale non è parte degli insegnamenti di Informatica L-31.
	}
		
	\end{QuestionList}
	
\end{document}
