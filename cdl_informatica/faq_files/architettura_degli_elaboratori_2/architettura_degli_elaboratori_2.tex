\documentclass{article}

\usepackage[utf8]{inputenc}
\usepackage[T1]{fontenc} 
\usepackage{hyperref}
\usepackage[dvipsnames]{color}
\usepackage[nodayofweek,ddmmyyyy]{datetime}
\usepackage{enumitem}

\newcommand{\domanda}[1]{\item{\ttfamily{\textsl{{\large{#1}}}}\\\\}}


\title{FAQ - \textbf{Architettura degli elaboratori II}}
\author{
	1$^{\circ}$ anno\\6 CFU, Secondo semestre\\
	Docente: Alberto Borghese\\ 
	\date{}
	\href{https://aborgheseae2.ariel.ctu.unimi.it/v5/home/Default.aspx}{Sito web}
}

\begin{document} 

\maketitle
	
\begin{QuestionList}
		
    \question{Dove si trova il sito web del corso?} {
        Si trova qui sopra se clicchi "sito web".
	}

		
	\question{Com'è strutturato l'esame?} {
		L’esame di teoria consiste in una parte scritta e orale. Quella scritta si svolge su piattaforma Exam.net e SEB (si presume a crocette), quella orale su zoom e ci si accede se si è passati la parte scritta.
		Per la parte di laboratorio invece è richiesto lo svolgimento di un progetto in assembly seguito poi da un colloquio orale. 
		Il voto di teoria vale 2/3 del voto finale, mentre quello di laboratorio 1/3.\\
		E' consigliato fare affidamento al sito del docente per eventuali cambiamenti in situazione covid.
	}
	
	\question{Qual è il materiale a disposizione per studiare?}{
		Sono disponibili le videolezioni sul sito ariel del docente, con i relativi pdf.
		Il libro consigliato dal docente è "Computer organization and design" di D. A. Patterson.
	}
		
		
	\end{QuestionList}
	
\end{document}
