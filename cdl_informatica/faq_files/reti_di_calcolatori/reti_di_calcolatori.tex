\documentclass{article}

\usepackage[utf8]{inputenc}
\usepackage[T1]{fontenc} 
\usepackage{hyperref}
\usepackage[dvipsnames]{color}
\usepackage[nodayofweek,ddmmyyyy]{datetime}
\usepackage{enumitem}

\newcommand{\domanda}[1]{\item{\ttfamily{\textsl{{\large{#1}}}}\\\\}}


\title{FAQ - \textbf{Reti di Calcolatori}}
\author{
	3$^{\circ}$ anno\\12 CFU, Primo semestre\\
	Docenti: Gian Paolo Rossi, Elena Pagani, Christian Quadri\\ 
	Siti web:
	\href{https://grossircc.ariel.ctu.unimi.it/v5/frm3/ThreadList.aspx?name=contenuti}{Teoria}
	\href{https://epaganircl.ariel.ctu.unimi.it/v5/frm3/ThreadList.aspx?name=contenuti}{Laboratorio}
	\date{}
}

\begin{document} 
	\maketitle
	
	\begin{QuestionList}
		
		\question{Dove si trova il sito web del corso?} {
		    Vi sono due siti per il corso, entrambi su Ariel; \href{https://grossircc.ariel.ctu.unimi.it/v5/frm3/ThreadList.aspx?name=contenuti}{il primo} riguarda la parte di teoria, mentre \href{https://epaganircl.ariel.ctu.unimi.it/v5/frm3/ThreadList.aspx?name=contenuti}{il secondo} quella di laboratorio. 
		
		    Fino all'AA 19-2020 il sito della parte di teoria si trovava a \href{http://old.nptlab.di.unimi.it/index.php/reti-di-calcolatori.html}{questo indirizzo}. 
	    }
		
		\question{Com'è strutturato l'esame?} {
		    L'esame è diviso in due parti: la prima parte consiste in uno scritto con domande inerenti gli argomenti trattati durante le lezioni di teoria e contribuisce al 60\% del voto finale; la seconda parte riguarda gli argomenti affrontati durante il laboratorio e contribuisce al 40\% del voto finale.
		
		    Quest'ultima è divisa in due prove, la prima riguarda la creazione di reti tramite software \href{https://en.wikipedia.org/wiki/Packet_Tracer}{CISCO Packet Tracer}, mentre la seconda riguarda la programmazione di reti tramite \href{https://docs.oracle.com/javase/8/docs/api/index.html?java/net/Socket.html}{socket Java}; il voto di questa seconda parte è dato dalla media delle due prove.
		
		    Il voto finale è dato dalla media pesata delle due parti; si accede alla seconda parte solo dopo aver superato la prima parte con un voto $\geq$ 13. 
		}
		
		\question{Qual è il materiale a disposizione per studiare?} {
		    Ci sono le videolezioni registrate, ed il libro di testo da integrare con le RFC disponibili sul \href{https://ietf.org/standards/rfcs/}{sito dell'IETF} (Internet Engineering Task Force).
		}
		
	\end{QuestionList}
	
\end{document}
